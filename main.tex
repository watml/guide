\documentclass{article}

% replace this with your conference style file
\usepackage{fullpage}
% to increase the number of fonts can be used
\newcommand\hmmax{0}
\newcommand\bmmax{0}
\usepackage{amsmath,amsthm,amssymb}

\usepackage[utf8]{inputenc}
\usepackage[english]{babel}

% use biblatex; it's 21 century!
% feel free to change some options
\usepackage[backend=biber,maxbibnames=5,style=numeric,maxcitenames=2,backref=false,uniquelist=false,uniquename=false,sorting=none,defernumbers=true,noerroretextools=true]{biblatex}
\usepackage{csquotes}
% change this to your bibfile
\addbibresource{main.bib}
% in case someone is used to natbib
\newcommand{\citep}{\parencite}
\newcommand{\citet}{\textcite}

\usepackage{xcolor}
\usepackage[colorlinks,linkcolor=black,citecolor=black,urlcolor=magenta,bookmarks=true]{hyperref}

% for easy cross-reference
\usepackage{cleveref}

% for better tables
\usepackage{booktabs}
% for controlling margins in enumerate/itemize
\usepackage{enumitem}

\usepackage{xspace}
\usepackage{graphicx,multirow}
% all figures go to the dedicated folder "figs"
\graphicspath{{figs/}}

% to draw pretty figures
\usepackage{tikz}
\usetikzlibrary{matrix,arrows,shadows,positioning,fadings,calc,shapes,fit}
\usepackage{pgfplots}

% allow equations to be numbered only when necessary 
% it is actually a good practice to number every equation (for the reader's sake)
\let\etoolboxforlistloop\forlistloop % save the good meaning of \forlistloop
\usepackage{autonum}
\let\forlistloop\etoolboxforlistloop % restore the good meaning of \forlistloop

% algorithm
\usepackage[linesnumbered,ruled,vlined]{algorithm2e}
\crefname{algocf}{algorithm}{algorithms}
\Crefname{algocf}{Algorithm}{Algorithms}
% if do not wish to number algorithms
%\renewcommand{\thealgocf}{}

% use todonotes for comments
\usepackage{todonotes}
\newcommand{\todoy}[2][]{\todo[size=\tiny,color=green!20!white,#1]{YY: #2}\xspace}


% define your own commands
\newcommand{\lb}{\mathtt{lb}}
\newcommand{\ub}{\mathtt{ub}}
\newcommand{\med}{\mathtt{med}}
\newcommand{\mi}{\mathtt{maxiter}}
\newcommand{\tol}{\mathtt{tol}}
\newcommand{\Fsf}{\mathsf{F}}

\title{How to write a \LaTeX\ draft}
\author{Yaoliang Yu}
\date{July 2020}

\begin{document}
	
\maketitle
	
\begin{abstract}
This is a sample \LaTeX\ file.
\end{abstract}
	
\section{Conventions}
\begin{itemize}[itemsep=0pt,topsep=0pt,leftmargin=*]
\item Figures and tables usually are placed at the top of each page. Ocassionally at the bottom.
\item Figure captions placed under the figure.
\item Table captions placed above the table. Some conferences do the opposite.
\end{itemize}
	
% change the option to [t] to place alg on top
\begin{algorithm}[H]
\DontPrintSemicolon
\KwIn{$u\in(0,1), \lb \leq \ub, \mi, \tol$}
		
\KwOut{$x$ such that $|\Fsf(x) - u| \leq \tol$}
		
\While{$\Fsf(\lb) > u$}{
$\ub \gets \lb$

$\lb \gets \lb /2$
}
		
\While{$\Fsf(\ub) < u$}{
$\lb \gets \ub$
			
$\ub \gets \ub * 2$
}
		
\For{$i=1, \ldots, \mi$ }{
			
$x \gets \tfrac{\lb+\ub}{2}$
			
$t \gets \Fsf(x)$
			
\If{ $t > u$}{$\ub \gets x$}
\Else{$\lb \gets x$}
			
\If{$ |t - u| \leq \tol $}{\textbf{break}}
}
		
\caption{Binary search for solving a monotonic nonlinear equation $\Fsf(x) = u$.}
\label{alg:bs}
\end{algorithm}
Let's cite the above algorithm using \verb|cleveref|: \Cref{alg:bs}.

Let's cite a paper: \parencite{Yu13}.

For writing a longer paper, need to split the main file into:
\begin{verbatim}
\input{introduction}

\input{background}

\input{main results}

\input{experiments}

\input{discussion}

\input{conclusion}

\input{appendix}
\end{verbatim}
	
\printbibliography[title={References}]
\end{document}
